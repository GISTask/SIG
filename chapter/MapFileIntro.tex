e\section{Introduction}

Untuk memanfaatkan fungsionalitasnya, MapServer akan membutuhkan sebuah mapfile. Mapfile merupakan file teks yang berekstensi .map, yang akan mendeskripsikan apa dan dimana sumber data dan bagaimana cara data tersebut ditampilkan. 
File Peta adalah text konfigurasi yang terdiri dari list setting yang digunakan untuk menggambar dan berinteraksi dengan peta. Informasi yang termuat didalamnya adalah layer data apa yang akan digambar, dimana focus geografis petanya, 
system proyeksi yang digunakan, format apa yang akan digunakan untuk menampilkan gambar, dan cara menentukan legenda dan skala pada peta.

\subsection(Tentang MapFile}

File .map juga bisa digunakan untuk.
\begin{itemize}
\item Debugging maps. ini adalah tipe dari plain text atau teks biasa yang menunjukan offset fungsi relatif untuk versi biner tertentu.

\subsection{ketentuan penulisan mapfile}
File yang disebut mapfile tersebut memiliki ketentuan-ketentuan penulisan sebagai berikut:
\begin{itemize}
\item Teks pada mapfile tidak bersifat case-sensitive.
\item Penulisan string yang berisi campuran beberapa karakter non-alphanumerik (selain karakter huruf dan angka) atau keywords milik MapServer harus diapit oleh tanda petik ganda ( “ ).
\item Setiap mapfile dapat digunakan untuk mendefinisikan (secara default) maksimal 50 layer peta.
\item Penulisan path lokasi file bisa dilakukan secara absolut maupun relatif.
\item Isi mapfile memiliki hierarki struktur dengan objek “MAP” sebagai “root”. Sementara objek-objek lainnya berada di bawah objek ini.
\item Pengguna dapat menambahkan baris-baris komentar di dalam mapfile dengan cara mengawali komentar tersebut dengan karakter pagar (#).

\section{Layer Object}

\subsection (CLASS)
CLASS adalah Sinyal awal dari kelas object. Di dalam lapisan, hanya dengan satu kelas aka digunakan untuk render fitur.  Setiap fitur diuji untuk setiap kelas dalam rangka dimana mereka didefinisikan dalam mapfile.  Kelas pertama yang tidak kalah dengan bi min / max skala ikatan dan ekspresi memeriksa untuk arus fitur akan digunakan untuk render.  Class terbagi antara dua yaitu :
